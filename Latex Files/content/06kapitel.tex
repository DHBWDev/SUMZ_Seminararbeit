%!TEX root = ../dokumentation.tex

%https://www.online-textbuero.de/953/wie-schreibe-ich-ein-fazit/

\chapter{Projektbezogener Vergleich der Webframeworks}
\section{Vergleichende Betrachtung}\label{sec:vergl}

Nachdem in den vorherigen drei Kapiteln die einzelnen Technologien detailliert beschrieben wurden, werden sie nun im folgenden Kapitel miteinander verglichen.

Das erste Vergleichskriterium bildet der Umfang an Funktionen, welche direkt im Framework integriert sind. Alle drei Webframeworks erlauben es, Daten komfortabel auf der Weboberfläche anzeigen. Jedoch bieten nur Angular und OpenUI5 die Möglichkeit im Standard, eine Datenquelle mittels „Two-Way-Binding“ anzubinden(siehe Abschnitt \ref{sec:aTemp} und Abschnitt \ref{sec:oDateien}). Bei React muss hierfür eine zusätzliche Komponente nachinstalliert werden(siehe Abschnitt \ref{sec:rWurz}). Ähnlich verhält es sich bei dem Routing zwischen mehreren Seiten der Anwendung. Angular und OpenUI5 liefern Möglichkeiten hierzu in ihrem Standard dazu aus, jedoch bei React muss hierfür wiederum eine zusätzliche Router-Implementierung eingebunden werden(siehe Abschnitt \ref{sec:rEntw} und \autocites[vgl.][8]{Zeigermann.2016}). 

Es gibt darüber hinaus weitere Unterschiede bei der empfohlenen Webarchitektur. Bei Angular und OpenUI5 wird man aufgrund des Frameworkaufbaus zur Verwendung des MVC-Modells gedrängt. Dies wird bei OpenUI5 deutlich, indem schon beim Einrichtungsassistenten die typische MVC-Struktur angelegt wird(siehe Abschnitt \ref{sec:oDateien}). Auch das Grundkonzept von Angular ist auf MVC ausgerichtet(siehe Abschnitt \ref{sec:aEinf}). Anders verhält es sich allerdings bei React, denn hier wird die Verwendung von Flux empfohlen, bei dem typischerweise die Daten nur in eine Richtung fließen(siehe Abschnitt \ref{sec:rEntw}). Dies generiert eine schnelle Performance bei React.

Ein weiteres wichtiges Merkmal bei Webframeworks ist die Datenanzeige. Bei OpenUI5 ist die Anbindung an einen Webserver über OData sehr einfach gehalten. Mit nur ein paar wenigen Codezeilen ist es möglich, Daten anzuzeigen und diese auch bei Bedarf zu ändern(siehe Listing \ref{lst:UI5View}). Jedoch ist man bei den Anzeigemöglichkeiten, welche einfach einzubinden sind, auf den von SAP entwickelten Standard beschränkt. Will man Daten anderweitig anzeigen muss man mit großem Aufwand andere Frameworks einbinden und den Datenaustausch mit diesen manuell implementieren. Bei Angular wird die Datenanzeige über Komponenten gesteuert, welche die Templates mit Inhalt füllen(siehe Abschnitt \ref{sec:aTemp}). Bei Angular sind oberflächenspezifische Anpassungen komfortabler und schneller möglich als bei OpenUI5, da diese direkt im CSS vorgenommen werden können. React hat bei der Datenanzeige klare Geschwindigkeitsvorteile im Vergleich zu OpenUI5 und Angular, da das Framework die Anzahl der DOM-Manipulationen auf ein Minimum reduziert(siehe Abschnitt \ref{sec:Rendern}). Daraus folgt, dass der Webbrowser weniger Aktionen ausführen muss. Im Vergleich zu Angular und React wird bei OpenUI5 das Design der Steuerelement vorgegeben und kann nur mit erheblichem Aufwand auf eigne Bedürfnisse angepasst werden(siehe Abschnitt \ref{sec:oDateien}).  

Bei der Wahl der Entwicklungsumgebung ist man bei Angular und React weitestgehend frei(siehe Abschnitt \ref{sec:aEntw} und \ref{sec:rEntw}). Man sollte jedoch bei beiden darauf achten, dass die Entwicklung in TypeScript unterstützt wird, falls man diese Programmiersprache verwendet. Bei OpenUI5 ist man in dieser Hinsicht beschränkter. Prinzipiell ist jeder Texteditor dafür geeignet, jedoch werden in Eclipse spezielle AddOns angeboten, welche das Erstellen und Entwickeln an einer Anwendung erleichtern(siehe Abschnitt \ref{sec:oEntw}).

Es gibt darüber hinaus auch Unterschiede bei den Wahlmöglichkeiten der Programmiersprache. Wie im Grundlagenkapitel \ref{sec:ts} erläutert bietet die von JavaScript abgeleitete Programmiersprache TypeScript zahlreiche Vorteile gegenüber ihrem Ursprung. Die Verwendung von TypeScript ist sowohl in Angular, als auch in React möglich(siehe Abschnitt \ref{sec:aEinf} und \ref{sec:rEinf}). Somit können mögliche Syntaxfehler schon beim Kompilieren erkannt werden und die Kompatibilität mit veralteten Webbrowsern sichergestellt werden. Bei OpenUI5 ist TypeScript zur Entwicklung nicht vorgesehen, deshalb wird hierbei das Entwickeln nur in der Webgrundsprache JavaScript empfohlen(siehe Anschnitt \ref{sec:oEinf}). Diese wird darüber hinaus auch von Angular sowie React unterstützt.

Ein weiteres Unterscheidungsmerkmal ist die Lizenzierung. Offiziell stehen alle drei zu vergleichenden Webframeworks unter Open-Source-Lizenzen. Jedoch werden lediglich die Frameworks Angular und React von einer breiten Entwicklergemeinde vorangetrieben. Bei OpenUI5 entwickelt hauptsächlich die SAP an Verbesserungen des Frameworks. Dies liegt daran, dass UI5 hauptsächlich in Unternehmen eingesetzt wird und diese ihre Entwicklungen nur selten veröffentlichen. 

Der letzte Unterschied zwischen den Webframeworks ist, dass sowohl Angular, als auch React native Versionen ihres Frameworks anbieten. Somit ist es möglich, mit ihnen native Apps für Android zu entwickeln. Bei OpenUI5 ist diese Möglichkeit nicht vorgesehen.


\section{Handlungsempfehlung}\label{sec:empf}

In diesem Abschnitt werden nun nach dem Vergleich aus dem vorherigen Kapitel \ref{sec:vergl} konkrete Empfehlungen zur Umsetzung in „business horizon“ ausgesprochen.

Um eine fundierte Empfehlung für eine Webtechnologie geben zu können muss zunächst die Anwendung auf ihre frontendrelevanten Eigenschaften untersucht werden. Nach einer umfassenden Analyse lassen sich einige Merkmale der Software erkennen. 
Sie umfasst viele unterschiedliche Komponenten und hat darüber hinaus noch einen Anmeldebildschirm. Daraus folgt eine Vielzahl von Views, zwischen denen gewechselt werden muss. Beim Routing hat sowohl Angular, als auch OpenUI5 seine Stärken. In React muss hierfür eine zusätzliche Komponente nachinstalliert werden.
In „business horizon“ gibt es ausschließlich unidirektionale Webserveranbindungen. OpenUI5 hätte bei bidirektionalen Bindungen seine Vorzüge, da es diese allerdings nur selten benötigt werden sind alle betrachteten Webframeworks in diesem Punkt gleichwertig.
Die Anwendung umfasst darüber hinaus viele nicht standardisiert aufgebaute Dialoge. Deshalb sollte das zu verwendende Framework viele Anpassungsmöglichkeiten der Anzeige bieten. Diese Möglichkeiten bieten Angular und React, bei OpenUI5 sind diese mit einem erheblichen Mehraufwand verbunden. Die von der SAP standardisierten Steuermöglichkeiten sind darüber hinaus nicht so eingängig als selbstentwickelte bei Angular und React. Denn beispielsweise gibt SAP im Standard lediglich neun Buttontypen vor\autocites[vgl.][]{UI5Doku1}.
Einen weiteren Vorteil für die Frameworks Angular und React stellt die breite Entwicklergemeinde dar. Denn es existieren zahlreiche Foren, in welchen über aufkommende Probleme beraten wird. UI5 wird zum Großteil von Unternehmen verwendet, falls diese auf etwaige Probleme stoßen wenden sie sich an den Support der SAP und teilen die Informationen nicht öffentlich im Internet. Somit würde man mit Angular und React während der Entwicklung einfacher zu Hilfen gelangen.
Ein weiteres Argument für Angular und React stellen vorhandene native Versionen dar. Falls im weiteren Projektverlauf eine App benötigt wird kann man auf diese Frameworks zurückgreifen.
Einen Nachteil für OpenUI5 stellen Komponenten dar, welche sich vom typischen HTML unterscheiden. Als Beispiel hierfür kann man die View heranziehen, welche in XML verfasst wird. Somit würde dies eine erhebliche Einstiegshürde für die Projektmitglieder bedeuten.
Eine weitere Einstiegshürde für das Projektteam wäre auch die Verwendung des Architekturmodells Flux, denn bisher wurde in den Vorlesungen ausschließlich das MVC-Modell gelehrt.
Das gewichtigste Argument für die Verwendung des Webframeworks Angular stellt das Faktum dar, dass bisher in der kompletten Anwendung „business horizon“ Angular verwendet wurde. Eine Änderung des Frameworks würde einen immensen Aufwand bedeuten. Bestärkt wird dieses Argument zusätzlich, da bei Angular bisher keinerlei Probleme aufgetreten sind.

Aufgrund der zuvor erörterten Argumente eignet sich Angular im Gesamten als Webfrontend für „business horizon“ am Besten. Es wird zudem empfohlen, dass der bestehende Code belassen wird, außer er bedarf einer schnellen Datenänderung, welche den Mehraufwand zur Implementierung von React rechtfertigt. Jedoch ist die Hürde, der Verwendung von React, bei Erweiterungen der Anwendung nicht so hoch und kann im Einzelfall in Betracht gezogen werden. OpenUI5 kommt für „business horizon“ nicht in Frage, da die zuvor erläuterten Vorteile nicht die Einstiegshürde für die Projektmitglieder rechtfertigt.



