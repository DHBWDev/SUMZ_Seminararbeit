%!TEX root = ../dokumentation.tex

\chapter{Beispiel Webframeworks}

\section{AngularJS}

\subsection{Allgemein}

\subsubsection{}


\subsubsection{Entwicklungsumgebung}
%% Hier besonders Angular Cli beschreiben


\subsection{Konzepte}
%% Beschreibung der Konzepte die in Angular verwendet werden
%% Hier kurz beschreiben wie die einzelnen Komponenten zusammenhängen

Beispiel Angular-Anwendung.

\begin{tabbing}
	mm \= mm \= mmmmmmmmmmmmmmmm \= \kill
	$\vdash$ \textbf{example/} \\ 
	| \> $\vdash$ \textbf{src/}\\ 
	| \> \> $\vdash$  \textbf{app/}\\
	| \> \>  --app.component.css\\
	| \> \>  --app.component.html\\
	| \> \>  --app.component.ts\\
	| \> \>  --app.module.ts\\
	| \> \> $\vdash$ \textbf{assets/} \\
	| \> \> $\vdash$ \textbf{environments/} \\
	| \> --index.html \\
	| \> --main.ts \\
	| \> --styles.css \\
	| \> --... \\
	| \> $\vdash$ \textbf{node\_modules/}\\ 
	| \> $\vdash$ \textbf{e2e/}\\   
	| --...\\
\end{tabbing}

%% Erklärug am Beispiel des Initialen Projekts --> Mit Hinweis darauf, dass dies Projekt beliebig erweitert werden kann.
\subsubsection{Module}


Eine Angular Anwendung ist modular aufgebaut und kann demnach aus mehreren Modulen bestehen. Ein Modul fasst eine zusammengehörige Codeeinheit zusammen. Ein Modul kann eine gewisse Funktionalität bereitstellen, die wiederum von anderen Modulen verwendet werden kann. \autocites[vgl.][103\psqq]{Steyer.2017} 

Die Module einer Angular-Anwendung können in Root-Module, Feature-Module und Shared-Module unterteilt werden. Das Root-Modul ist einmalig in einer Angular-Anwendung vorhanden und wird beim Starten der Anwendung aufgerufen. Die Anwendung wird durch dieses Modul konfiguriert. Mithilfe von Feature Modulen kann eine Anwendung nach Anwendungsfällen gruppiert werden. Shared-Module fassen die Teile zusammen, die unabhängig vom Anwendungsfall verwendet werden können. \autocites[vgl.][528\psqq]{Freeman.2018}[vgl.][]{Google.c}[vgl.][105\psqq]{Steyer.2017}

In der Beispielanwendung stellt die Klasse \textit{AppModule} siehe \autoref{lst_RootModule} das Root-Modul dar. Ein Modul wird durch den Derokator \textit{@NgModule} gekennzeichnet. Ein Modul kann verschiedene weitere Module über die Eigenschaft \textit{import} importieren und damit die bereitgestellten Funktionalitäten verwenden. Die Eigenschaft \textit{declarations} 

\begin{lstlisting}[caption=Das Root-Module in der Datei app.module.ts, label=lst_RootModule, language=Java]
import { BrowserModule } from '@angular/platform-browser';
import { NgModule } from '@angular/core';

@NgModule({
	imports: [ BrowserModule],
	declarations: [AppComponent],
	bootstrap: [AppComponent]
})
export class AppModule { }
\end{lstlisting}




\subsubsection{Komponenten und Templates}

%Beschreibung des Starts einer Angular Anwedung 

%% Komponenten sind Direktiven 
%% Template und Komponente beschreiben
Komponenten sind Klassen, die Daten und Logik für die zugehörigen Templates bereitstellen. Diese ermöglichen die Aufteilung einer Angular Anwendung in logisch getrennte Teile. \autocite[vgl.][401]{Freeman.2018} 

Eine Komponente wird durch den Dekorator \textit{@Component} gekennzeichnet und kann über verschiedene Dekorator-Eigenschaften (auch: Metadaten) konfiguriert werden. Die Eigenschaft \textit{selector} identifiziert das HTML-Element, dass durch diese Komponente repräsentiert wird. Zur Anzeige der bereitgestellten Daten kann entweder ein Inline-Template \textit{template} definiert oder auf ein externes Template \textit{templateUrl} verwiesen werden. \autocites[vgl.][]{Google.b}[vgl.][405]{Freeman.2018}[vgl.][47\psqq]{Steyer.2017} 

Für weitere Dekorator-Eigenschaften wird auf \textcite[405]{Freeman.2018} verwiesen. Ein Beispiel für die Implementierung einer Komponente findet sich in \autoref{Angular-Komponente}.

\begin{lstlisting}[caption=Implementierung einer Komponente, label=Angular-Komponente, language=Java]
import {Component} from '@angular/core';

@Component({
	selector: hello,
	template: `<h1>Hello World</h1>`
})
export class HelloWorldComponent {
}
\end{lstlisting}


Zur Darstellung von Komponenten nutzt Angular Templates. Ein Template besteht aus HTML Code erweitert um Angular Ausdrücke. Das Template kann Pipes, weitere Komponenten, Data-Binding-Ausdrücke oder Direktiven enthalten. \autocites[vgl.][]{Google.b}[vgl.][52]{Steyer.2017} 

Data-Bindings stellen eine Beziehung zwischen den Daten der Komponente und einem HTML-Element her. Hierdurch kann das Aussehen, der Inhalt oder das Verhalten dieses Elements dynamisch verändert werden. \autocites[vgl.][237\psqq]{Freeman.2018}[vgl.][52\psq]{Steyer.2017}

Es können drei Arten von Bindings unterschieden werden. Die Unterscheidung erfolgt mittels der Fließrichtung der Daten.
\begin{description}
	\item [Two-Way-Binding \lbrack ()\rbrack{} ] 
	\item [Property-Binding \lbrack \rbrack{} ] 
	\item [Event-Binding ()] 
\end{description}

\begin{tabular}{lcl}
	Two-Way-Binding&\lbrack (target)\rbrack{}&aaa\\ 
	Property-Binding&\lbrack target\rbrack{}&aaaa\\ 
	Event-Binding&(target)&aaaa\\ 
\end{tabular} 


\subsubsection{Direktiven}

%% Attribut und Struktur Direktive Beschreiben

Mit Direktiven kann einem Element zusätzliches Verhalten hinzugefügt werden. \autocite[vgl.][265]{Steyer.2017} In Angular werden folgende drei Arten von Direktiven unterschieden. \autocite[vgl.]{Google.}

\begin{itemize}
	\item Komponenten
	\item Attribut-Direktiven
	\item strukturelle Direktiven 
\end{itemize}

Angular stellt Direktiven zur Verfügung (engl. Built-In Directives). Diese können durch eigene Direktiven erweitert werden. \autocite[vgl.][261]{Freeman.2018}

Mit strukturellen Direktiven kann der Inhalt des HTML-Dokuments angepasst werden, indem Elemente dem diesem hinzugefügt oder entnommen werden. Hierfür verwenden die strukturellen Direktiven Templates, die beliebig oft gerendert werden. \autocites[vgl.][269\psqq]{Steyer.2017}[vgl.][365]{Freeman.2018}

Beispiele für strukturellen Direktiven aus AngularJS \autocite[vgl.][261\psqq]{Freeman.2018}:
\begin{description}
	\item [ngIf] Fügt dem HTML-Dokument Inhalt hinzu, wenn die Bedingung wahr ist. 
	\item [ngfor] Fügt für jedes Item einer Datenquelle den gleichen Inhalt dem HTML-Dokument hinzu.
	\item [ngSwitch] Fügt dem HTML-Dokument, abhängig vom Wert eines Ausdrucks, Inhalt hinzu.
\end{description} 

Mit Attribut-Direktiven kann das Verhalten und Aussehen des zugehörigen Elementes angepasst werden, indem Attribute hinzugefügt oder entfernt werden. \autocite[vgl.][339]{Freeman.2018} 

Beispiele für Attribut-Direktiven aus Angular-JS \autocite[vgl.][249\psqq]{Freeman.2018}:
\begin{description}
	\item [ngStyle] Mit dieser Direktive können unterschiedliche Style-Eigenschaften dem Element hinzugefügt werden.
	\item [ngClass] Weißt dem Element mehrere Klassen hinzu. 
\end{description}

Komponenten sind Direktiven mit einer eigenen View. \autocites[vgl.][265]{Steyer.2017}[vgl.][401]{Freeman.2018} Aufgrund der hohen Bedeutung der Komponente in AngularJS, wird auf diese im nächsten Gliederungspunkt näher eingegangen.

\subsubsection{Services}

%% Was bringen Services? Wie funktionieren diese?
%% Kommunkikation mit dem DB-Server über Services

%%\subsubsection{Pipes} 
%% ???


%%\subsubsection{Dependency Injection}
%% Was ermöglicht dieses Konzept? --> leichteres testen, lose Kopplung



\subsection{Verwendung}

\subsubsection{Einordnung in den Kontext}


\section{ReactJS}

\subsection{Allgemein}
%Entwicklung
%Kompatibilität
%Lizenz

%ReactJs ist kein MVC Framework! Welches Problem löst ReactJS?
% Kurz auf Flux eingehen????

ReactJS ist ein von Entwicklern des Unternehmens Facebook Inc. entwickeltes JavaScript Framework. \autoref{Gackenheimer.2015} 


%Löst bestimmte Probleme
%



\subsection{Konzepte}

\subsubsection{Components}

%Klasse und Methode
%Lifecycle einer Klasse
%Vorteile einer Klasse (State versus Props)

%Kurz auf JSX eingehen

\subsubsection{Lifecycle}

\subsubsection{Virtual DOM}

\subsection{Verwendung}




\section{OpenUI5}