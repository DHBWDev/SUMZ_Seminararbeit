%!TEX root = ../dokumentation.tex

%https://www.online-textbuero.de/953/wie-schreibe-ich-ein-fazit/

\chapter{Fazit}

Im Zuge dieser Seminararbeit sollten die Webfrontendtechnologien Angular, ReactJS sowie OpenUI5 mit dem Ziel analysiert und verglichen werden, um zu überprüfen, ob Angular das passendste Framework für die Anwendung „business horizon“ darstellt. 

In der Analyse und dem Vergleich konnten verschiedenste Stärken und Schwächen der einzelnen Technologien erarbeitet werden. Es wurde festgestellt, dass OpenUI5 seine Stärken in der bidirektionalen Bindung zu einem Webservice und seine Schwächen bei der Anpassung auf spezifische Anforderungen hat. Die Vorzüge von ReactJS sind beispielsweise die hohe Performance bei Änderungen von Inhalten auf der Weboberfläche. Jedoch hat ReactJS beim Routing zwischen verschiedenen Webansichten (Views) gravierende Nachteile. Angular erfüllt alle anwendungsrelevanten Leistungsmerkmale und stellt keine gravierende Beeinträchtigung von benötigten Funktionen dar. 

Somit wurde das Ziel dieser Seminararbeit erfüllt und mit der Verwendung von Angular eine fundierte Handlungsempfehlung ausgesprochen.
