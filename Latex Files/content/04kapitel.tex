%!TEX root = ../dokumentation.tex
\chapter{ReactJS}
\label{ch:reactJS}

\section{Allgemein}

\subsection{Einführung in das Framework}

ReactJS ist eine von Facebook entwickelte JavaScript Bibliothek zur Entwicklung von Benutzeroberflächen. Im Gegensatz zu Angular ist ReactJS ein reines View-Framework. Das Framework wird unter anderem bei Facebook, Instagram, Netflix, Airbnb und dem Content Management System Wordpress eingesetzt. Es bietet einige Vorteile bei der Entwicklung von Anwendungen mit großen Benutzeroberflächen mit Daten, die sich häufig verändern. 

Das Framework ist in JavaScript geschrieben und kann mit JavaScript oder einer in JavaScript übersetzbare Sprache wie TypeScript verwendet werden.\autocites[vgl.][1\psqq]{Gackenheimer.2015}[vgl.][3\psqq]{Zeigermann.2016}

\comment{An dieser Stelle eventuell noch mehr auf das Problem eingehen.}

\subsection{Vorbereitung der Entwicklungsumgebung}



\section{Konzepte}

\subsection{Virtueller DOM}

%ReactElement

\subsection{Komponenten}
Komponenten sind das zentrale Element in ReactJS. Sie enthalten sowohl die Logik als auch den zugehörigen Teil der Benutzeroberfläche. Eine Komponente kann entweder als Klasse oder als Funktion (engl. functional components) implementiert werden. Wobei der Zustand (State), Lifecycle-Methoden und refs sich nicht  mit Funktionskomponenten verwenden lassen. Laut \textcite[vgl.][82\psq]{Zeigermann.2016} solle man Komponenten vorzugsweise als Funktion implementieren,  außer es soll eine nur von klassenbasierten Komponenten bereitgestelltes Feature verwendet werden. 

Die Funktion muss genau ein React-Element zurückgeben. Die implementierte Komponente trägt den gleichen Namen wie die Funktion. Die Funktion Hello in \autoref{lst:KomponenteFunktion} implementiert die Komponente Hello.

\begin{lstlisting}[caption=Beispiel einer Komponente als Funktion, label=lst:KomponenteFunktion, language=Java]
import React from 'react';
funtion Hello(){
	return <h1>Hello World</h1>;
}
\end{lstlisting}

Eine Klasse, die eine React-Komponente implementiert, muss von \textit{React.Component} erben. Zudem muss die Klasse eine Methode \textit{render()}, die ein React-Element zurückgibt, implementieren. Das \autoref{lst:KomponenteKlasse} zeigt die Implementierung der Komponente Hello als Klasse.

\begin{lstlisting}[caption=Beispiel einer Komponente als Klasse, label=lst:KomponenteKlasse, language=Java]
import React from 'react';
class Hello extends React.Component{
	render() {
		return <h1>Hello World</h1>;
	}
}
\end{lstlisting}

%Beschreibung von Eigenschaften

Einer Komponente können beim Erzeugen Properties übergeben werden. Diese sind allerdings innerhalb der Komponente unveränderlich.\autocites[vgl.][71\psqq]{Zeigermann.2016}[vgl.][11\psqq]{Stefanov.2017}

%Beschreibung des Zustands

%Beschreibung des Lebenszyklusses einer Komponente

  
%Auf die Zusammenarbeit der einzelnen Komponenten eingehen...Wie sieht eine typische React Anwendung aus?

\section{Verwendung}


