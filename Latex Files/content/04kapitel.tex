%!TEX root = ../dokumentation.tex
\chapter{ReactJS}

\section{Allgemein}

\subsection{Einführung in das Framework}

ReactJS ist eine von Facebook entwickelte JavaScript Bibliothek zur Entwicklung von Benutzeroberflächen. Im Gegensatz zu Angular ist ReactJS ein reines View-Framework. Das Framework wird unter anderem bei Facebook, Instagram, Netflix, Airbnb und dem Content Management System Wordpress eingesetzt. Es bietet einige Vorteile bei der Entwicklung von Anwendungen mit großen Benutzeroberflächen mit Daten, die sich häufig verändern. 

Das Framework ist in JavaScript geschrieben und kann mit JavaScript oder einer in JavaScript übersetzbare Sprache wie TypeScript verwendet werden.\autocites[vgl.][1\psqq]{Gackenheimer.2015}[vgl.][3\psqq]{Zeigermann.2016}

\comment{An dieser Stelle eventuell noch mehr auf das Problem eingehen.}

\subsection{Vorbereitung der Entwicklungsumgebung}



\section{Konzepte}

\subsection{Die Spracherweiterung JSX}

\subsection{Komponenten}

%Klasse und Methode
%Lifecycle einer Klasse
%Vorteile einer Klasse (State versus Props)

%Kurz auf JSX eingehen

\section{Verwendung}


