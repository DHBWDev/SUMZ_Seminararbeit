%!TEX root = ../dokumentation.tex
%https://www.online-textbuero.de/902/wie-schreibe-ich-eine-einleitung/

\chapter{Einleitung}
\vskip -3em
\textbf{Autor: Sebastian Greulich}

\section{Projektziel}

Im Zuge des Projektes „Stochastische Unternehmensbewertung mittels Methoden der Zeitreihenanalyse“, kurz SUMZ, soll die Software „business horizon“ weiterentwickelt werden. Diese Software bewertet auf Basis von betriebswirtschaftlichen Daten den zukünftigen Unternehmenswert. Bisher wird dabei zur Oberflächendarstellung das Webfrontendframework Angular verwendet.

Diese Seminararbeit soll mit Hilfe eines Vergleichs mit ReactJS und OpenUI5 bewerten, ob Angular aus verschiedenen Gesichtspunkten die passendste Oberflächentechnologie für das Projekt darstellt. Im Zuge dieser wissenschaftlichen Arbeit soll daher eine Handlungsempfehlung für das zukünftige Webfrontendframework ausgesprochen werden.

\section{Aufbau}

Zunächst werden die technischen Grundlagen zu den vergleichenden Technologien behandelt. Hierbei wird beispielsweise auf die verwendeten Programmiersprachen JavaScript und TypeScript sowie auf verschiedene Webarchitekturen eingegangen. 

Im weiteren Verlauf der vorliegenden Seminararbeit werden die einzelnen clientseitigen Webframeworks AngularJS, ReactJS und OpenUI5 beschrieben. In den Beschreibungen werden die allgemeinen Rahmenbedingungen der einzelnen Frameworks dargelegt. Dies sind beispielsweise die verwendete Lizenz oder auch die Anforderungen an die Entwicklungsumgebung. Darüber hinaus wird in jeder Beschreibung die Architektur des Frameworks beleuchtet. Hierzu gehören sowohl die notwendigen Dateikomponenten als auch das Definieren und Reagieren auf Events. Zum Abschluss jeder Beschreibung wird auf die möglichen Einsatzgebiete der Technologien eingegangen.

Nach den detaillierten Beschreibungen der Webframeworks wird in \autoref{sec:empf} erörtert, welche Technologie sich am besten zur Verwendung in „business horizon“ eignet. Durch diesen Schritt wird das Projekt damit abgerundet, dass als Fazit eine Handlungsempfehlung ausgesprochen werden kann.