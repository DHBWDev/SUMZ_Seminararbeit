%!TEX root = ../dokumentation.tex
%https://www.online-textbuero.de/902/wie-schreibe-ich-eine-einleitung/

\chapter{Einleitung}

\section{Projektumfeld}

Im Zuge des Projektes „Stochastische Unternehmensbewertung mittels Methoden der Zeitreihenanalyse“, kurz SUMZ, soll die Software „business horizon“ weiterentwickelt werden. Mit dieser Anwendung soll es möglich werden, auf Basis von betriebswirtschaftlichen Daten den zukünftigen Unternehmenswert zu bewerten. In „business horizon“ wird momentan das Webfrontendframework AngularJS verwendet. Diese Seminararbeit soll in einem Vergleich mit ReactJS und OpenUI5 bewerten, ob AngularJS aus verschiedenen Gesichtspunkten die passendste Oberflächentechnologie für das Projekt darstellt. Im Zuge dieser wissenschaftlichen Arbeit soll eine Handlungsempfehlung für das zukünftige Webfrontendframework ausgesprochen werden.

\section{Aufbau}

Im folgenden Kapitel „Webframeworks“ sollen die technischen Grundlagen zum Verständnis der zu vergleichenden Technologien gelegt werden. Hierbei wird beispielsweise auf die verwendeten Programmiersprachen JavaScript und TypeScript, aber auch auf verschiedene Möglichkeiten zur Quellcodestrukturierung eingegangen. 

Im weiteren Verlauf der vorliegenden Seminararbeit werden die einzelnen clientseitigen Webframeworks AngularJS, ReactJS und OpenUI5 beschrieben. In den Beschreibungen werden die allgemeinen Rahmenbedingungen der einzelnen Frameworks dargelegt. Diese bilden beispielsweise die verwendete Lizenz oder auch die Browserkompatibilität. Darüber hinaus wird in jeder Beschreibung die Architektur des Frameworks beleuchtet. Hierzu gehören sowohl die notwendigen Dateikomponenten, als auch das Definieren und Reagieren auf Events. Zum Abschluss jeder Beschreibung wird auf die möglichen Einsatzgebiete der Technologien eingegangen.

Nach den detaillierten Beschreibungen der Webframeworks wird in Kapitel \ref{sec:empf} erörtert, welche Technologie sich am besten zur Verwendung in „business horizon“ eignet.