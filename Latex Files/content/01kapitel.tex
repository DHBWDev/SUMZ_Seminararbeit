%!TEX root = ../dokumentation.tex
%https://www.online-textbuero.de/902/wie-schreibe-ich-eine-einleitung/

\chapter{Einleitung}

\section{Projektumfeld}

Im Zuge des Projektes „Stochastische Unternehmensbewertung mittels Methoden der Zeitreihenanalyse“ soll die Software „business horizon“ weiterentwickelt werden. Mit dieser Software soll es möglich werden, auf Basis von betriebswirtschaftlichen Daten den zukünftigen Unternehmenswert zu bewerten. In „business horizon“ wird momentan das Webfrontendframework AngularJS verwendet. Diese Seminararbeit soll in einem Vergleich mit ReactJS und OpenUI5 bewerten, ob AngularJS aus verschiedenen Gesichtspunkten die  beste Oberflächentechnologie für das Projekt darstellt. 

\section{Aufbau}

Im darauffolgenden Kapitel 2 „Webframeworks“ werden die technischen Grundlagen zum besseren Verständnis der zu vergleichenden Technologien  gelegt. Hierbei wird beispielsweise auf die verwendeten Programmiersprachen JavaScript und TypeScript, aber auch auf verschiedene Möglichkeiten zur Quellcodestrukturierung eingegangen. 

Im weiteren Verlauf der vorliegenden Seminararbeit werden die einzelnen clientseitigen Webframeworks AngularJS, ReactJS und OpenUI5 beschrieben. In den Beschreibungen wird auf die allgemeinen Rahmenbedingungen der einzelnen Frameworks eingegangen. Diese bilden beispielsweise die verwendete Lizenz oder auch die Browserkompatibilität. Darüber hinaus wird in jeder Beschreibung die Architektur des Frameworks beleuchtet. Hierzu gehören sowohl die notwendigen Dateikomponenten, als auch das Definieren und Reagieren auf Events. Zum Abschluss jeder Beschreibung wird auf die möglichen Einsatzgebiete der Technologien eingegangen.

Nach den detaillierten Beschreibungen der Webframeworks wird in Kapitel „xy“ erörtert, welche Technologie sich am besten die Software „business horizon“ eignet.