%!TEX root = ../dokumentation.tex

\chapter{Beispiel Webframeworks}

\section{AngularJS}

\subsection{Allgemein}

\subsubsection{}


\subsubsection{Entwicklungsumgebung}
%% Hier besonders Angular Cli beschreiben


\subsection{Konzepte}
%% Beschreibung der Konzepte die in Angular verwendet werden
%% Hier kurz beschreiben wie die einzelnen Komponenten zusammenhängen


\subsubsection{Direktiven}

%% Attribut und Struktur Direktive Beschreiben

Mit Direktiven kann einem Element zusätzliches Verhalten hinzugefügt werden. \autocite[vgl.][265]{Steyer.2017} In Angular werden folgende drei Arten von Direktiven unterschieden. \autocite[vgl.]{Google.}

\begin{itemize}
	\item Komponenten
	\item Attribut-Direktiven
	\item strukturelle Direktiven 
\end{itemize}

Angular stellt Direktiven zur Verfügung (engl. Built-In Directives). Diese können durch eigene Direktiven erweitert werden. \autocite[vgl.][261]{Freeman.2018}

Mit strukturellen Direktiven kann der Inhalt des HTML-Dokuments angepasst werden, indem Elemente dem diesem hinzugefügt oder entnommen werden. Hierfür verwenden die strukturellen Direktiven Templates, die beliebig oft gerendert werden. \autocites[vgl.][269\psqq]{Steyer.2017}[vgl.][365]{Freeman.2018}

Beispiele für strukturellen Direktiven aus AngularJS \autocite[vgl.][261\psqq]{Freeman.2018}:
\begin{description}
	\item [ngIf] Fügt dem HTML-Dokument Inhalt hinzu, wenn die Bedingung wahr ist. 
	\item [ngfor] Fügt für jedes Item einer Datenquelle den gleichen Inhalt dem HTML-Dokument hinzu.
	\item [ngSwitch] Fügt dem HTML-Dokument, abhängig vom Wert eines Ausdrucks, Inhalt hinzu.
\end{description} 

Mit Attribut-Direktiven kann das Verhalten und Aussehen des zugehörigen Elementes angepasst werden, indem Attribute hinzugefügt oder entfernt werden. \autocite[vgl.][339]{Freeman.2018} 

Beispiele für Attribut-Direktiven aus Angular-JS \autocite[vgl.][249\psqq]{Freeman.2018}:
\begin{description}
	\item [ngStyle] Mit dieser Direktive können unterschiedliche Style-Eigenschaften dem Element hinzugefügt werden.
	\item [ngClass] Weißt dem Element mehrere Klassen hinzu. 
\end{description}

Komponenten sind Direktiven mit einer eigenen View. \autocites[vgl.][265]{Steyer.2017}[vgl.][401]{Freeman.2018} Aufgrund der hohen Bedeutung der Komponente in AngularJS, wird auf diese im nächsten Gliederungspunkt näher eingegangen.

\subsubsection{Komponenten und Templates}

%% Komponenten sind Direktiven 
%% Template und Komponente beschreiben
Komponenten sind Klassen, die Daten und Logik für die zugehörigen Templates bereitstellen. Diese ermöglichen eine Angular Anwendung in verschiedene logisch getrennte Teile aufzuteilen. \autocite[vgl.][401]{Freeman.2018} 

Eine Komponente wird durch den Dekorator \textit{@Component} gekennzeichnet und kann über verschiedene Dekorator-Eigenschaften (auch: Metadaten) konfiguriert werden. Die Eigenschaft \textit{selector} identifiziert das HTML-Element, dass durch diese Komponente repräsentiert wird. Zur Anzeige der bereitgestellten Daten kann entweder ein Inline-Template \textit{template} definiert oder auf ein externes Template \textit{templateUrl} verwiesen werden. \autocites[vgl.]{Google.b}[vgl.][405]{Freeman.2018}[vgl.][47\psqq]{Steyer.2017} Für weitere Dekorator-Eigenschaften wird auf \textcite[405]{Freeman.2018} verwiesen.



\begin{lstlisting}[caption=Implementierung einer Komponente, label=Angular-Komponente, title=Implementierung einer Komponente,language=JavaScript]
@Component({
	selector: 
})
export class ExampleComponent {
}
\end{lstlisting}



\subsubsection{Services}

%% Was bringen Services? Wie funktionieren diese?
%% Kommunkikation mit dem DB-Server über Services

\subsubsection{Pipes} 
%% ???


%%\subsubsection{Dependency Injection}
%% Was ermöglicht dieses Konzept? --> leichteres testen, lose Kopplung



\subsection{Verwendung}

\subsubsection{Einordnung in den Kontext}


\section{ReactJS}

\subsection{Allgemein}
%Entwicklung
%Kompatibilität
%Lizenz

%ReactJs ist kein MVC Framework! Welches Problem löst ReactJS?
% Kurz auf Flux eingehen????

ReactJS ist ein von Entwicklern des Unternehmens Facebook Inc. entwickeltes JavaScript Framework. \autoref{Gackenheimer.2015} 


%Löst bestimmte Probleme
%



\subsection{Konzepte}

\subsubsection{Components}

%Klasse und Methode
%Lifecycle einer Klasse
%Vorteile einer Klasse (State versus Props)

%Kurz auf JSX eingehen

\subsubsection{Lifecycle}

\subsubsection{Virtual DOM}

\subsection{Verwendung}




\section{OpenUI5}