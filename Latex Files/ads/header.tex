
%!TEX root = ../dokumentation.tex

%
% Nahezu alle Einstellungen koennen hier getaetigt werden
%

\RequirePackage[l2tabu, orthodox]{nag}	% weist in Commandozeile bzw. log auf veraltete LaTeX Syntax hin

\documentclass[%
	pdftex,
	oneside,			% Einseitiger Druck.
	12pt,				% Schriftgroesse
	parskip=half,		% Halbe Zeile Abstand zwischen Absätzen.
%	topmargin = 10pt,	% Abstand Seitenrand (Std:1in) zu Kopfzeile [laut log: unused]
	headheight = 12pt,	% Höhe der Kopfzeile
%	headsep = 30pt,	% Abstand zwischen Kopfzeile und Text Body  [laut log: unused]
	%headsepline,		% Linie nach Kopfzeile.
	%footsepline,		% Linie vor Fusszeile.
	footheight = 16pt,	% Höhe der Fusszeile
	abstracton,		% Abstract Überschriften
	DIV=calc,		% Satzspiegel berechnen
%BCOR=8mm,		% Bindekorrektur links: 8mm
	headinclude=false,	% Kopfzeile nicht in den Satzspiegel einbeziehen
	footinclude=false,	% Fußzeile nicht in den Satzspiegel einbeziehen
	listof=totoc,		% Abbildungs-/ Tabellenverzeichnis im Inhaltsverzeichnis darstellen
	toc=bibliography,	% Literaturverzeichnis im Inhaltsverzeichnis darstellen
]{scrreprt}	% Koma-Script report-Klasse, fuer laengere Bachelorarbeiten alternativ auch: scrbook

% Einstellungen laden
\usepackage{xstring}
\usepackage[utf8]{inputenc}
\usepackage[T1]{fontenc}
\sloppy %für korrekten Zeilenumbruch

\newcommand{\einstellung}[1]{%
  \expandafter\newcommand\csname #1\endcsname{}
  \expandafter\newcommand\csname setze#1\endcsname[1]{\expandafter\renewcommand\csname#1\endcsname{##1}}
}
\newcommand{\langstr}[1]{\einstellung{lang#1}}

\einstellung{matrikelnr}
\einstellung{titel}
\einstellung{kurs}
\einstellung{datumAbgabe}
\einstellung{firma}
\einstellung{firmenort}
\einstellung{abgabeort}
\einstellung{abschluss}
\einstellung{studiengang}
\einstellung{dhbw}
\einstellung{betreuer}
\einstellung{gutachter}
\einstellung{zeitraum}
\einstellung{arbeit}
\einstellung{autor}
\einstellung{sprache}
\einstellung{schriftart}
%\einstellung{seitenrand}
\einstellung{seitenrandLinks}
\einstellung{seitenrandRechts}

\einstellung{kapitelabstand}
\einstellung{spaltenabstand}
\einstellung{zeilenabstand}
\einstellung{zitierstil}
 % verfügbare Einstellungen
%%%%%%%%%%%%%%%%%%%%%%%%%%%%%%%%%%%%%%%%%%%%%%%%%%%%%%%%%%%%%%%%%%%%%%%%%%%%%%%
%                                   Einstellungen
%
% Hier können alle relevanten Einstellungen für diese Arbeit gesetzt werden.
% Dazu gehören Angaben u.a. über den Autor sowie Formatierungen.
%
%
%%%%%%%%%%%%%%%%%%%%%%%%%%%%%%%%%%%%%%%%%%%%%%%%%%%%%%%%%%%%%%%%%%%%%%%%%%%%%%%


%%%%%%%%%%%%%%%%%%%%%%%%%%%%%%%%%%%% Sprache %%%%%%%%%%%%%%%%%%%%%%%%%%%%%%%%%%%
%% Aktuell sind Deutsch und Englisch unterstützt.
%% Es werden nicht nur alle vom Dokument erzeugten Texte in
%% der entsprechenden Sprache angezeigt, sondern auch weitere
%% Aspekte angepasst, wie z.B. die Anführungszeichen und
%% Datumsformate.
\setzesprache{de} % oder en
%%%%%%%%%%%%%%%%%%%%%%%%%%%%%%%%%%%%%%%%%%%%%%%%%%%%%%%%%%%%%%%%%%%%%%%%%%%%%%%%

%%%%%%%%%%%%%%%%%%%%%%%%%%%%%%%%%%% Angaben  %%%%%%%%%%%%%%%%%%%%%%%%%%%%%%%%%%%
%% Die meisten der folgenden Daten werden auf dem
%% Deckblatt angezeigt, einige auch im weiteren Verlauf
%% des Dokuments.
\setzematrikelnr{}
\setzekurs{WWI16B2}
\setzetitel{Clientseitige Webframeworks wie Angular, ReactJS und OpenUI5}
\setzedatumAbgabe{07.01.2019}
\setzefirma{}
\setzefirmenort{}
\setzeabgabeort{Karlsruhe}
\setzeabschluss{Bachelor of Science (B.Sc.)}
\setzestudiengang{Wirtschaftsinformatik}
\setzedhbw{Karlsruhe}
\setzebetreuer{Betreuer}
\setzegutachter{Prof. Dr. Ratz, Dietmar \\  \> Prof. Dr. Pohl, Philipp\\  \> Schulmeister-Zimolong, Dennis}
\setzezeitraum{}
\setzearbeit{Seminararbeit}
\setzeautor{Sebastian Greulich, Fabio Krämer}
%%%%%%%%%%%%%%%%%%%%%%%%%%%%%%%%%%%%%%%%%%%%%%%%%%%%%%%%%%%%%%%%%%%%%%%%%%%%%%%%

%%%%%%%%%%%%%%%%%%%%%%%%%%%% Literaturverzeichnis %%%%%%%%%%%%%%%%%%%%%%%%%%%%%%
%% Bei Fehlern während der Verarbeitung bitte in ads/header.tex bei der
%% Einbindung des Pakets biblatex (ungefähr ab Zeile 110,
%% einmal für jede Sprache), biber in bibtex ändern.
\newcommand{\ladeliteratur}{%
%%\addbibresource{bibliographie.bib}
\addbibresource{Literatur.bib}
}
%% Zitierstil
%% siehe: http://ctan.mirrorcatalogs.com/macros/latex/contrib/biblatex/doc/biblatex.pdf (3.3.1 Citation Styles)
%% mögliche Werte z.B numeric-comp, alphabetic, authoryear
%%\setzezitierstil{authortitle-dw}
\setzezitierstil{authoryear-icomp}
%%%%%%%%%%%%%%%%%%%%%%%%%%%%%%%%%%%%%%%%%%%%%%%%%%%%%%%%%%%%%%%%%%%%%%%%%%%%%%%%

%%%%%%%%%%%%%%%%%%%%%%%%%%%%%%%%% Layout %%%%%%%%%%%%%%%%%%%%%%%%%%%%%%%%%%%%%%%
%% Verschiedene Schriftarten
% laut nag Warnung: palatino obsolete, use mathpazo, helvet (option scaled=.95), courier instead
\setzeschriftart{lmodern} % palatino oder goudysans, lmodern, libertine

%% Paket um Textteile drehen zu können
%\usepackage{rotating}
%% Paket um Seite im Querformat anzuzeigen
%\usepackage{lscape}

%% Seitenränder
%\setzeseitenrand{2.5cm}
\setzeseitenrandLinks{3.5cm}
\setzeseitenrandRechts{2.5cm} 

%% Abstand vor Kapitelüberschriften zum oberen Seitenrand
\setzekapitelabstand{0pt}

%% Spaltenabstand
\setzespaltenabstand{10pt}
%%Zeilenabstand innerhalb einer Tabelle
\setzezeilenabstand{1.5}
%%%%%%%%%%%%%%%%%%%%%%%%%%%%%%%%%%%%%%%%%%%%%%%%%%%%%%%%%%%%%%%%%%%%%%%%%%%%%%%%

%%%%%%%%%%%%%%%%%%%%%%%%%%%%% Verschiedenes  %%%%%%%%%%%%%%%%%%%%%%%%%%%%%%%%%%%
%% Farben (Angabe in HTML-Notation mit großen Buchstaben)
\newcommand{\ladefarben}{%
	\definecolor{LinkColor}{HTML}{00007A}
	\definecolor{ListingBackground}{HTML}{FCF7DE}
}
%% Mathematikpakete benutzen (Pakete aktivieren)
%\usepackage{amsmath}
%\usepackage{amssymb}

%% Programmiersprachen Highlighting (Listings)
\newcommand{\listingsettings}{%
	\lstset{%
		language=Java,			% Standardsprache des Quellcodes
		numbers=left,			% Zeilennummern links
		stepnumber=1,			% Jede Zeile nummerieren.
		numbersep=5pt,			% 5pt Abstand zum Quellcode
		numberstyle=\tiny,		% Zeichengrösse 'tiny' für die Nummern.
		breaklines=true,		% Zeilen umbrechen wenn notwendig.
		breakautoindent=true,	% Nach dem Zeilenumbruch Zeile einrücken.
		postbreak=\space,		% Bei Leerzeichen umbrechen.
		tabsize=2,				% Tabulatorgrösse 2
		basicstyle=\ttfamily\footnotesize, % Nichtproportionale Schrift, klein für den Quellcode
		showspaces=false,		% Leerzeichen nicht anzeigen.
		showstringspaces=false,	% Leerzeichen auch in Strings ('') nicht anzeigen.
		extendedchars=true,		% Alle Zeichen vom Latin1 Zeichensatz anzeigen.
		captionpos=b,			% sets the caption-position to bottom
		backgroundcolor=\color{ListingBackground}, % Hintergrundfarbe des Quellcodes setzen.
		xleftmargin=0pt,		% Rand links
		xrightmargin=0pt,		% Rand rechts
		frame=single,			% Rahmen an
		frameround=ffff,
		rulecolor=\color{darkgray},	% Rahmenfarbe
		fillcolor=\color{ListingBackground},
		keywordstyle=\color[rgb]{0.133,0.133,0.6},
		commentstyle=\color[rgb]{0.133,0.545,0.133},
		stringstyle=\color[rgb]{0.627,0.126,0.941}
	}
}
%%%%%%%%%%%%%%%%%%%%%%%%%%%%%%%%%%%%%%%%%%%%%%%%%%%%%%%%%%%%%%%%%%%%%%%%%%%%%%%%

%%%%%%%%%%%%%%%%%%%%%%%%%%%%%%%% Eigenes %%%%%%%%%%%%%%%%%%%%%%%%%%%%%%%%%%%%%%%
%% Hier können Ergänzungen zur Präambel vorgenommen werden (eigene Pakete, Einstellungen)



 % lese Einstellungen

\newcommand{\iflang}[2]{%
  \IfStrEq{\sprache}{#1}{#2}{}
}

\langstr{abkverz}
\langstr{anhang}
\langstr{glossar}
\langstr{deckblattabschlusshinleitung}
\langstr{artikelstudiengang}
\langstr{studiengang}
\langstr{anderdh}
\langstr{von}
\langstr{dbbearbeitungszeit}
\langstr{dbmatriknr}
\langstr{dbkurs}
\langstr{dbfirma}
\langstr{dbbetreuer}
\langstr{dbgutachter}
\langstr{sperrvermerk}
\langstr{erklaerung}
\langstr{abstract}
\langstr{listingname}
\langstr{listlistingname}
\langstr{listingautorefname}
 % verfügbare Strings
\input{lang/\sprache} % Übersetzung einlesen

% Einstellung der Sprache des Paketes Babel und der Verzeichnisüberschriften
\iflang{de}{\usepackage[english, ngerman]{babel}}
\iflang{en}{\usepackage[ngerman, english]{babel}} 


%%%%%%% Package Includes %%%%%%%

%\usepackage[margin=\seitenrand,foot=1cm]{geometry}	% Seitenränder und Abstände
\usepackage[left = \seitenrandLinks, right = \seitenrandRechts, top = 2.5cm, bottom = 2cm, foot = 0.75cm, head = 0.75cm ]{geometry}	
\usepackage[activate]{microtype} %Zeilenumbruch und mehr
\usepackage{setspace}
\setstretch{1.5}
\usepackage{makeidx}
\usepackage[autostyle=true,german=quotes]{csquotes}
\usepackage{longtable}
\usepackage{enumitem}	% mehr Optionen bei Aufzählungen
\usepackage{graphicx}
\usepackage{pdfpages}   % zum Einbinden von PDFs
\usepackage{xcolor} 	% für HTML-Notation
\usepackage{float}
\usepackage{array}
\usepackage{calc}		% zum Rechnen (Bildtabelle in Deckblatt)
\usepackage[right]{eurosym}
\usepackage{wrapfig}
\usepackage{pgffor} % für automatische Kapiteldateieinbindung
\usepackage[perpage, hang, multiple, stable]{footmisc} % Fussnoten
\usepackage[printonlyused, footnote]{acronym} % falls gewünscht kann die Option footnote eingefügt werden, dann wird die Erklärung nicht inline sondern in einer Fußnote dargestellt
\usepackage{listings}

\usepackage{setspace} % Hiermit kann der Zeilenabstand für einen  Bereich festgelegt werden

% Notizen. Einsatz mit \todo{Notiz} oder \todo[inline]{Notiz}. 
\usepackage[obeyFinal,backgroundcolor=yellow,linecolor=black]{todonotes}
% Alle Notizen ausblenden mit der Option "final" in \documentclass[...] oder durch das auskommentieren folgender Zeile
% \usepackage[disable]{todonotes}

% Kommentarumgebung. Einsatz mit \comment{}. Alle Kommentare ausblenden mit dem Auskommentieren der folgenden und dem aktivieren der nächsten Zeile.
\newcommand{\comment}[1]{\par {\bfseries \color{blue} #1 \par}} %Kommentar anzeigen
%\newcommand{\comment}[1]{} %Kommentar ausblenden


%%%%%% Configuration %%%%%

%% Anwenden der Einstellungen

\usepackage{\schriftart}
\ladefarben{}

% Titel, Autor und Datum
\title{\titel}
\author{\autor}
\date{\datum}

% PDF Einstellungen
\usepackage[%
	pdftitle={\titel},
	pdfauthor={\autor},
	pdfsubject={\arbeit},
	pdfcreator={pdflatex, LaTeX with KOMA-Script},
	pdfpagemode=UseOutlines, 		% Beim Oeffnen Inhaltsverzeichnis anzeigen
	pdfdisplaydoctitle=true, 		% Dokumenttitel statt Dateiname anzeigen.
	pdflang={\sprache}, 			% Sprache des Dokuments.
]{hyperref}

% (Farb-)einstellungen für die Links im PDF
\hypersetup{%
	colorlinks=true, 		% Aktivieren von farbigen Links im Dokument
	linkcolor=LinkColor, 	% Farbe festlegen
	citecolor=LinkColor,
	filecolor=LinkColor,
	menucolor=LinkColor,
	urlcolor=LinkColor,
	linktocpage=true, 		% Nicht der Text sondern die Seitenzahlen in Verzeichnissen klickbar
	bookmarksnumbered=true 	% Überschriftsnummerierung im PDF Inhalt anzeigen.
}
% Workaround um Fehler in Hyperref, muss hier stehen bleiben
\usepackage{bookmark} %nur ein latex-Durchlauf für die Aktualisierung von Verzeichnissen nötig

% Schriftart in Captions etwas kleiner
\addtokomafont{caption}{\small}

% Literaturverweise (sowohl deutsch als auch englisch)
\iflang{de}{%
\usepackage[
	backend=biber,		% empfohlen. Falls biber Probleme macht: bibtex
	bibwarn=true,
	bibencoding=utf8,	% wenn .bib in utf8, sonst ascii
	sortlocale=de_DE,
	style=\zitierstil,
	maxcitenames=2, 
	isbn=false,
	]{biblatex}
}
\iflang{en}{%
\usepackage[
	backend=bibtex,		% empfohlen. Falls biber Probleme macht: bibtex
	bibwarn=true,
	bibencoding=utf8,	% wenn .bib in utf8, sonst ascii
	sortlocale=en_US,
	style=\zitierstil,
]{biblatex}
}

\DefineBibliographyStrings{ngerman}{andothers={et\ al\adddot}} % aus u.a. zu et al. machen

\ladeliteratur{}

% Glossar
%\usepackage[nonumberlist,toc]{glossaries}

%%%%%% Additional settings %%%%%%

% Hurenkinder und Schusterjungen verhindern
% http://projekte.dante.de/DanteFAQ/Silbentrennung
\clubpenalty = 10000 % schließt Schusterjungen aus (Seitenumbruch nach der ersten Zeile eines neuen Absatzes)
\widowpenalty = 10000 % schließt Hurenkinder aus (die letzte Zeile eines Absatzes steht auf einer neuen Seite)
\displaywidowpenalty=10000

% Bildpfad
\graphicspath{{images/}}

% Einige häufig verwendete Sprachen
%%\lstloadlanguages{PHP,Python,Java,C,C++,bash}
\lstdefinelanguage{JavaScript}{
	keywords={break, case, catch, continue, debugger, default, delete, do, else, finally, for, function, if, in, instanceof, new, return, switch, this, throw, try, typeof, var, void, while, with},
	morecomment=[l]{//},
	morecomment=[s]{/*}{*/},
	morestring=[b]',
	morestring=[b]",
	sensitive=true
}

\lstloadlanguages{SQL, JavaScript, bash}
\listingsettings{}
% Umbennung des Listings
\renewcommand\lstlistingname{\langlistingname}
\renewcommand\lstlistlistingname{\langlistlistingname}
\def\lstlistingautorefname{\langlistingautorefname}

% Abstände in Tabellen
\setlength{\tabcolsep}{\spaltenabstand}
\renewcommand{\arraystretch}{\zeilenabstand}

%Quelle unter dem Bild
\newcommand*{\quelle}[1]{\par\raggedleft\footnotesize Quelle:~#1}

\usepackage[singlespacing=true]{scrlayer-scrpage} %Paket für die Manipulation von Kopf und Fußzeile
\usepackage{caption}%Paket ist dazu da, damit die Abbildungen aus dem Anhang nicht im Abbildungsverzecihnis angezeigt werden.
\usepackage{subcaption}


\counterwithout{figure}{chapter} %durchgängige Nummerierung der Abbildungen